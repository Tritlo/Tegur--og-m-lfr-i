\documentclass[12pt]{book}
\usepackage[utf8]{inputenc}

% not needed with polyglossia
\usepackage[utf8]{inputenc}
\usepackage[T1]{fontenc}

%xe/lualatex
%\usepackage{polyglossia}
%\setdefaultlanguage{icelandic}


\usepackage{graphics,amsmath,amsfonts,amsbsy,amssymb,amsthm}
\usepackage{fancyvrb}
\usepackage[a4paper]{geometry}
\usepackage{graphicx}
\usepackage{hyperref}
\usepackage{datatool}
\usepackage{float}
\usepackage{mdframed}
\usepackage{listingsutf8}
\usepackage{enumerate}
\usepackage{comment}
\usepackage{epstopdf}
\usepackage{caption}
\usepackage{subcaption}
\usepackage{titling}
\usepackage{tikz}
\usepackage[shortlabels]{enumitem}
\usepackage{mathtools}
\usepackage{tabu}

\usepackage{accents}

\setlength{\parskip}{8pt plus 1pt minus 1pt}
%Verdur ad vera her, sumir pakkar dependa a thetta.
\usepackage[icelandic]{babel}

\newcommand{\subtitle}[1]{%
  \posttitle{%
    \par\end{center}
    \begin{center}\large#1\end{center}
    \vskip0.5em}%
}
%viljum ekki númeraða kafla á dæmum



\newcommand{\nonums}{\setcounter{secnumdepth}{-1}}

%flýtiskipanir
\newcommand{\e}{\emph}


%\newcommand{\R}{\Real}
%\newcommand{\C}{\Complex}
%\newcommand{\Z}{\Integer}
%\newcommand{\N}{\Natural}
%\newcommand{\Q}{\Rational}
\newcommand{\R}{\mathbb{R}}
\newcommand{\X}{\mathbb{X}}
\newcommand{\Y}{\mathbb{Y}}
\newcommand{\K}{\mathbb{K}}
\newcommand{\C}{\mathbb{C}}
\newcommand{\Con}{\mathcal{C}}
\newcommand{\Z}{\mathbb{Z}}
\newcommand{\N}{\mathbb{N}}
\newcommand{\Q}{\mathbb{Q}}
\newcommand{\f}{\frac}
\newcommand{\1}{\frac{1}}
\newcommand{\eps}{\f{\epsilon}}
\newcommand{\Lra}{\Leftrightarrow}
\newcommand{\Th}{\text{ þegar }}
\newcommand{\Ef}{\text{ ef }}
\newcommand{\Og}{\text{ og }}


\newcommand{\inner}[1]{\accentset{\circ}{#1}}
\newcommand{\eR}{\widetilde{\R}}

\newcommand{\sumninfty}[1]{\sum_{n = {#1}}^{\infty}}
\newcommand{\sumoinfty}{\sum_{n = 1}^{\infty}}
\newcommand{\sumzinfty}{\sum_{n = 0}^{\infty}}

\newcommand{\com}[1]{\set{\text{#1}}}
\newcommand{\Com}[1]{\set{\text{Athsmd: \text{#1}}}}

\newcommand{\ub}[2]{\underbrace{#1}_{\text{#2}}}
\newcommand{\ubt}[2]{$\ub{\text{#1}}{#2}$}


\newenvironment{inum}{\begin{enumerate}[label=(\roman*).]}{\end{enumerate}}
\newenvironment{anum}{\begin{enumerate}[label=(\alph*).]}{\end{enumerate}}


\newcommand{\bcondef}{\left\{ \begin{array}{l l}}
\newcommand{\abs}[1]{\left|#1\right|}

\newcommand{\econdef}{\end{array} \right.}
\DeclarePairedDelimiter{\condef}{\bcondef}{\econdef}

\DeclarePairedDelimiter{\ceil}{\lceil}{\rceil}
\DeclarePairedDelimiter{\floor}{\lfloor}{\rfloor}
\DeclarePairedDelimiter{\set}{\{}{\}}
\DeclarePairedDelimiter{\braket}{\langle}{\rangle}


\newenvironment{lausn}{\begin{proof}[Lausn]}{\end{proof}}

\newcommand{\sep}{\;|\;}

\newcommand{\fig}[2]{
\begin{figure}[H]
  \centering
  \includegraphics{#1}
  \caption{#2}
  \label{fig:#1}
\end{figure}
}


\newtheorem*{setn}{Setning}
\newtheorem*{hsetn}{Hjálparsetning}
\lstset{  literate={á}{{\'a}}1
                  {ó}{{\'o}}1
                  {ú}{{\'u}}1
                  {ð}{{\dh}}1
                  {í}{{\'i}}1
                  {é}{{\'e}}1
                  {ö}{{\"o}}1
                  {þ}{{\th}}1
                  {æ}{{\ae}}1
                  {ý}{{\'y}}1
                  {Á}{{\'A}}1
                  {Ó}{{\'O}}1
                  {Ú}{{\'U}}1
                  {Ð}{{\DH}}1
                  {Í}{{\'I}}1
                  {É}{{\'E}}1
                  {Ö}{{\"O}}1
                  {Þ}{{\TH}}1
                  {Æ}{{\AE}}1
                  {Ý}{{\'Y}}1}


\theoremstyle{definition}
\newtheorem*{skgr}{Skilgreining}
\newtheorem*{daemi}{Dæmi}
\newtheorem*{frumsenda}{Frumsenda}

\theoremstyle{remark}
\newtheorem*{ath}{Athugasemd}
\newtheorem*{innsk}{Innskot}

\title{Tegur- og Málfræði}
\subtitle{Fyrirlestrarnótur}
\author{Matthías Páll Gissurarson}
\date{Vor 2015}

\newcommand{\cP}{\mathcal{P}}
\newcommand{\cF}{\mathcal{F}}
\newcommand{\cS}{\mathcal{S}}
\newcommand{\cM}{\mathcal{M}}



\begin{document}

\maketitle

\tableofcontents

\chapter{2015-01-05}
\section{Mengi: Ritháttur og upprifjun}
Gefum okkur að til grundvallar liggi ``hæfilega''  stórt almengi.

\begin{itemize}
\item Fjölskylda af hlutmengjum í mengi M er vörpun \(a: \Lambda \to \mathcal{P}(M) \). Skrifum í stað $a(\alpha)$ og táknum $a$ með $(A_{\alpha})_{\alpha \in \Lambda}$. $\Lambda$ kallast \emph{stikamengi} fjölskyldunnar.
\item \[ \bigcap_{\alpha \in \Lambda} := \set{x \in M | x \in A_{\alpha} \text{ f. \emph{öll} $\alpha$ úr $\Lambda$}}\]
\item \[ \bigcup_{\alpha \in \Lambda} := \set{x \in M | x \in A_{\alpha} \text{ f. \emph{eitthvert} $\alpha$ úr $\Lambda$}}\]

\item Fyrir $A \subseteq M$ setjum við \[A^C = M \setminus A := \set{ x \in M | x \not\in A}\] og köllum \emph{fyllimengi} A (í M).
\item Fyrir $ A, B \subseteq M$ setjum við \[B \setminus A := \set{ x \in B | x \not\in A} = B \cap A^C\]
\item Fyrir $A,B \subseteq M$ kallast \[ A \Delta B := ( A \setminus B) \cup (B \setminus A) \] (s.s. bara þau stök sem eru í öðru hvoru, en ekki báðum) \emph{samhverfur mismunur} A og B.

\item \emph{Reglur de Morgan} \[ (\bigcup_{\alpha \in \Lambda} A_{\alpha})^C = \bigcap_{\alpha \in \Lambda} A_{\alpha}^C\]

\[ (\bigcap_{\alpha \in \Lambda} A_{\alpha})^C = \bigcup_{\alpha \in \Lambda} A_{\alpha}^C\]
\item Fyrir $ A \subseteq M$ skgr. við kennifall $A$:
\[ 1_A: M \to \R, 1_A(x) = \bcondef 1 & \Ef x \in A \\ 0 & \Ef x \not\in A \econdef \]A

\item Fyrir $A,B \subseteq M$ gildir $1_{A \cap B} = 1_A \cdot 1_B$, $1_{A \cup B} = 1_A + 1_B - 1_{A \cap B}$ og $1_{A^C} = 1 - 1_A$.
\end{itemize}

\section{Firðrúm}
Hlutmengi í firðrúmi er opið \emph{þþaa} það sé sammengi af opnum kúlum. Sér í lagi er hlutmengi í $\R$ opið \emph{þþaa} það sé sammengi af opnum bilum.

\section{Riemann Darboux- heildið}
er mjög þunglamalegt og hegðar sér illa m.t.t markgilda; runa af Riemann heildanlegum föllum getur hæglega stefnt (í sérhverjum punkti) á fall sem er ekki Riemann heildanlegt (stundum kallað einfaldlega samleitið).

\begin{daemi}
\[f: [0,1] \to R, f(x) = \bcondef 0 & \Ef x \not\in \Q \\ 1 & \Ef x \in \Q \econdef  | f = 1_{Q \cap [0,1]}. \]

$\Q$ er teljanlegt, svo að til er gagntækt fall $q: \N \to \Q \cap [0,1]$, þ.e.a.s. til er runa $(q_n)_{n \in \N}$ þar sem sérhver ræð tala úr $[0,1]$ kemur nkvl. einu sinni fyrir. Setjum $A_n = \set{q_1, \dotsc, q_n}$ og $f_n = 1_{A_n}$. Þá er $f_n \nearrow f$ og ljóst er að $f_n$-in eru Riemann heildanleg, en $f$ ekki.

Hugmyndin er sú að stækka flokk heildanlegra falla þ.a. ekki $f$ gildi.

Allar eðlilegar reiknireglur varðandi heildi gilda áfram.

Flokkurinn er mun þjálli m.t.t. ýmissa aðgerða, sérstaklega markgildistöku.
\end{daemi}

\section{Núllmengi}
\subsection{Dæmi}
\begin{enumerate}[i)]
\item Öll teljanleg hlutmengi í $\R$ eru Núllmengi: Ef $(x_n)_{n \geq 1}$ er upptalning á stökum teljanlegs mengis, þá er $A = \cup_{n \geq 1} [x_n, x_n]$ og $\sum_{n = 1}^{\infty} \ell([x_n,x_n]) = \sum_{n=1}^{\infty} 0 = 0$
\item $\ell(\emptyset) = \ell(]0,0[) = 0$, svo að $\emptyset$ er núllmengi.
\end{enumerate}

\section*{Innskot}

\begin{setn}
Látum $\sum_{n=1}^{\infty}a_n$ vera alsamleitna tvinntalnaröð og $\sigma: \overbrace{\N}^{n \geq 1} \to \N$ vera gagntæka vörpun (umröðun). Þá er röðin $\sumninfty{1} a_{\sigma(n)}$ alsamleitin og $\sumninfty{1} a_{\sigma(n)} = \sumoinfty a_n$.
\end{setn}
\begin{proof}
Fyrir $N > 0$ er til $M > 0$ þ.a. $\set{\sigma(1), \dotsc, \sigma(n)} \subseteq \set{1, \dotsc, M}$.
\[\sum_{n=1}^{N}  \abs{a_{\sigma(n)}}
\leq \sum_{n = 1}^M \abs{a_n}
\ub{\leq \sumninfty{1} \abs{a_n}}{$< +\infty$}
\]
þar með er $\sumoinfty \abs{a_{\sigma(n)}}$ alsamleitin.
Setjum nú fyrir öll $n \geq 1$ $S_n := \sum_{k=1}^n a_k$ og $T_n := \sum_{k=1}^n a_{\sigma(k)}$.
Okkur nægir að sýna að f. sérhvert $\epsilon > 0$ sé til $n_0$ þ.a. $\abs{S_n - T_n} \leq \epsilon$ f. öll $n \geq n_0$.

Gefum okkur $\epsilon > 0$ og veljum $N$ þ.a. $\sum_{j = N+1}^{\infty} \abs{a_j} \leq \epsilon$. Tökum svo $n_0$ þ.a. $ \set{1, \dotsc, n} \subseteq \set{\sigma(1), \dotsc, \sigma(n_0)}$. Þá styttast tölurnar $\abs{a_n, \dotsc, a_n}$ allar út í mismuninum $S_n - T_n$, og þar með $\abs{S_n - T_n} \leq \sum_{j = N+1}^{\infty} \abs{a_j}  \leq \epsilon$ ef $n \geq n_0$.
\end{proof}

\begin{setn}[Umröðunarsetning Riemanns]
Ef $\sum a_n$, þar sem $a_n$ eru rauntölur er samleitin, en $\sum \abs{a_n} = \infty$, þ.e.
samleitin en ekki alsamleitin, og $L \in [-\infty, +\infty]$, þá er til gagntæk vörpun
$\sigma: \N \to \N$ sem hefur þann eiginleika að $\sumoinfty a_{\sigma(n)} = L$.
\end{setn}
\begin{proof} Sleppt.

Lausleg útskýring:
\[a_n^+ \bcondef a_n & \Ef a_n \geq 0 \\ 0 & \Ef a_n < 0 \econdef 
a_n^- \bcondef \abs{a_n} & \Ef a_n \leq 0 \\ 0 & \Ef a_n > 0 \econdef
\]
Ef $a_n$ er alsamleitin, þá er $\sum a_n = \ub{\sum a_n^+}{saml} - \ub{\sum a_n^-}{saml}$,
en ef $\sum a_n$ er samleitin en ekki alsamleitin, þá eru $\sum a_n^+$ og $\sum a_n^-$ báðar ósamleitnar.

\end{proof}

Látum nú $\Lambda$ vera eitthvað teljanlegt mengi og $(a_{\alpha})_{\alpha \in \Lambda}$ vera fjölskylda af tvinntölum. Við segjum að summan $\sum_{\alpha \in \Lambda}$ sé alsamleitin ef
til er gagntæk vörpun $\sigma: \N \to \Lambda$ þ.a. $\sumoinfty \abs{a_{\sigma(n)}} < +\infty$ (þ.e.a.s. $\sumoinfty \abs{a_{\sigma(n)}}$ er alsamleitin). Táknum þá summuna með $\sumoinfty \abs{a_n}$. Skv. setn sem við vorum að sanna, þá er þessi eiginleiki óháður valinu á $\sigma$.


\section*{Æfing}

Látum $(a_{\alpha})_{\alpha \in \Lambda}$ eins og áður. Sýnið: Ef $a_{\alpha}$ er alsamleitin, þá gildir f. sérhvert $\epsilon > 0$ að til er endanlegt hlutmengi $I$ í $\Lambda$ sem uppfyllir
\[ \abs{\sum_{\alpha \in \Lambda} a_{\alpha} - \sum_{\alpha \in J} } < \epsilon \]
fyrir öll endanleg $J$ þ.a. $I \subseteq J \subseteq \Lambda $. 

\begin{ath} Út frá umröðunarsetningu fæst að samleitin röð $\sumoinfty a_n$ sem breytist ekki við umraðanir er alsamleitin. 

%{} skilgreinir lokun, þ.a. inní henni get ég deffað eins og ég vil.
{ 
    \def\aal{a_{\alpha}}
    \def\aall{(\aal)_{\alpha \in \Lambda}}
    \def\saal{\sum_{\alpha \in \Lambda} a_{\alpha}} 

\newcommand{\sal}[1]{\sum_{\alpha \in #1} a_{\alpha } }

Takið eftir að summa 
$\saal$ er alsamleitin þþaa til sé $ K > 0$ sem uppfyllir

$\sal{I} \abs{\aal} \leq K$ f. öll endanleg $I \subseteq \Lambda$.

\def\mn{(m,n)}
\def\amn{a_{m,n}}
Lítum nú á tilfellið $(\amn)_{\mn \in \underbrace{\N \times \N}_{\N^2}}$.

Ef $\sum_{\mn \in \N^2} \amn$ er alsamleitin, þá er til $k > 0$ þ.a.
$\sum_{m=1}^M \sum_{n=1}^N | \amn| \leq k$ f. öll $M$ og $N$.

Þá er $\sum_{n=1}^N \abs{\amn} \leq k$ f. öll m og öll $N$. og því
$\sumoinfty \abs{\amn} \leq k$ og $\forall m$ og þar með $\sum_{n=1}^N \left(\sum_{m=1}^M \abs{\amn}\right) \leq k$ f. öll $M$. Af því leiðir að $\sum_{m=1}^{\infty} \left( \sumoinfty \abs{\amn} \right) \leq k$ og þar með er $\sum_{m=1}^{\infty} \left( \sumoinfty \abs{\amn} \right)$ alsamleitin.
}

\end{ath}

\section{Fyrri fyrirlestur}

\subsection{}
\begin{proof}

Gefum okkur $\epsilon > 0$. Fyrir hvert m veljum við runu $(I^n_k)_{k \geq 1}$ af bilum sem uppfyllir

\[N_n \subset \bigcup_{k \geq 1} I_k^n \text{ og } \sumninfty{k} \ell(I^n_k) < \eps{2^n}\]

Veljum gagntæka vörpun $\sigma: \N \to \N^2$ og skrifum $(I^n_k)_{(m,n) \in \N^2}$ sem runu
$(J_j)_{j \in \N}$ með hjálp $\sigma$, þ.e.a.s $J_j = I^n_k$

þþaa $\sigma (j) = (m,n)$.

\begin{ath}
$N \subseteq \R$ er núllmengi ef fyrir sérhvert $\epsilon > 0$ er til runa $(I_n)_{n}$ af bilum þ.a.
$N \subseteq \bigcup I_n$ og $\sum_{n \geq 1} \ell(I_n) < \epsilon$.

\end{ath}
\end{proof}
\chapter{2015-01-07}

\section*{Innskot}

{

\def\mn{(m,n)}
\def\amn{a_{m,n}}
\def\summn{\sum_{\mn \in \N^2}}

Þá gildir um öll $M$ og $N$ að $\sum_{m=1}^{\infty} \sum_{n=1}^{\infty} \abs{\amn} \leq \summinfty (\sumoinfty \abs{\amn})$, svo að $\sum_{\mn \in \N^2} \amn$ er alsamleitin.

Sýnum að þá sé $\summinfty ( \sumoinfty \amn ) = \summn \amn$.

Nú: Gefum okkur $\epsilon > 0$. Veljum M svo stórt að $\summinfty ( \sumoinfty \amn ) < \eps{4}$.

Veljum svo $N$ það stórt að $\sum_{n = N+1}^{\infty} \abs{\amn} < \eps{2^{n+2}}$ fyrir $1 \leq m \leq M$.


Við gefum 

\[\abs{\sum_{(m,n) \in \N^2} \sum_{m=1}^M \sum_{n=1}^N \amn } < \eps{2} \]

þá fæst: 
\begin{gather*}
\abs{\summinfty ( \sumoinfty \amn) - \sum_{(m,n) \in \N^2} \amn}  \leq \abs{\summinfty ( \sumoinfty \amn) - \sum_{m=1}^{M} \sum_{n=1}^N \amn} + \underbrace{\abs{\sum_{m=1}^M \sum_{n=1}^N \amn - \sum_{(m,n) \in \N^2} \amn}}_{< \eps{2}}\\
< \abs{\summinfty ( \sumoinfty \amn)} + \abs{\sum_{m=1}^M ( \sum_{n = N+1}^{\infty} \amn)} + \eps{2}\\
< \eps{4} + \sum^n \eps{2^{m+2}} + \eps{2} < \eps{4} + \eps{4} + \eps{2} = \epsilon
\end{gather*}

Hér líkur innskoti.

}

\section{}

\begin{proof}[Sönnun á 2.1.1]

Gefum okkur $\epsilon > 0$. Fyrir hvert m veljum við runu $(I^n_k)_{k \geq 1}$ af bilum sem uppfyllir

\[N_n \subset \bigcup_{k \geq 1} I_k^n \text{ og } \sumninfty{k} \ell(I^n_k) < \eps{2^n}\]

Veljum gagntæka vörpun $\sigma: \N \to \N^2$ og skrifum $(I^n_k)_{(m,n) \in \N^2}$ sem runu
$(J_j)_{j \in \N}$ með hjálp $\sigma$, þ.e.a.s $J_j = I^n_k$

þþaa $\sigma (j) = (m,n)$.

\begin{ath}
$N \subseteq \R$ er núllmengi ef fyrir sérhvert $\epsilon > 0$ er til runa $(I_n)_{n}$ af bilum þ.a.
$N \subseteq \bigcup I_n$ og $\sum_{n \geq 1} \ell(I_n) < \epsilon$.

\end{ath}

Ljóst er að $N = \bigcup_{n \geq 1} N_n \subseteq \bigcup_{(m,n) \in \N^2} I^n_k = \bigcup_{j \geq 1} J_j$
og skv. innskotinu hér að framan fæst þá að
$\sumninfty{j} \ell(J_j) = \sumoinfty ( \sumninfty{k} \ell(I_k^n)) < \sumninfty{n} \eps{2^n} = \epsilon$

\end{proof}

\begin{daemi}

Öll teljanleg hlutmengi í $\R$ eru núll mengi, en Cantor-mengið er dæmi um óteljanlegt núllmengi (sjá bls. 19 í bók og ``mengi og firðrúm'').
\end{daemi}


\section{}
\begin{ath}2.2.1:
F. öll $A \subseteq \R$ gildir
\begin{itemize}
\item $m^*(A) \geq 0$
  
\item $Z_A$ er af gerðinni $]r, \infty]$ eða $[r,\infty]$.
\end{itemize}
\end{ath}
\begin{proof}[Sönnun á setn 2.2.2]
A er núllmengi þþaa fyrir sérhver $\epsilon > 0$ sé til runa af bilum $(I_n)_{n \geq 1}$ þ.a. $A \subset \bigcup_{n \geq 1} I_n$ og $\sum_{n \geq 1} \ell(I_n) < \epsilon$ þþaa
fyrir sérhvert $\epsilon > 0$ er til tala $x$ úr $Z_A$ þ.a. $x < \epsilon$ þþaa $m^*(A) = \inf{Z_A} = 0$
\end{proof}

\begin{proof}[Sönnun a setn 2.2.3] Sérhver run af bilum sem þekur $B$ þekur líka $A$ svo að $Z_A \subseteq Z_B$ og þar með $m^{*}(A) = \inf{Z_A} \leq \inf{Z_B} = m^{*}$
\end{proof}

\section*{Æfing}
\begin{enumerate}[(a)]
    \item Látum $I_1, \dotsc, I_n$ vera rauntalna bil sem uppfylla $[a,b] \subseteq I_1 \cup \dotsb \cup I_n$. Sýnið að $b - a \leq \ell(I_1) + \dotsb + \ell(I_n)$.
    \item Látum $(I_n)_{n \geq 1}$ vera runu af bilum sem þekur óendanlegt bil I. Sýnið að $\sum_{n \geq 1} \ell(I_n) = \infty$.
\end{enumerate}


\section*{}
\begin{proof}[Sönnun á 2.2.4]
    Skv. (b) í æfingunni er niðurstaðan rétt ef $I$ er ótakmarkað. Gerum því ráð fyrir að $\ell(I) < \infty$. Byrjum á tilfellinu $I = [a,b]$

    \begin{ath}
        Ljóst er að $m^{*}(I) \leq \ell(I)$ vegna þess að bilarunan $I_1 := I$ og $I_k = [0,0]$
        fyrir $k \geq 2$ þekur $I$ og $\sumninfty{k} \ell(I_k) = \ell(I) + 0 + \dotsb + 0 = \ell(I)$.
        
        Sönnum því $\ell(I) \leq m^{*} (I)$.
    \end{ath}
    Gefum okkur $\epsilon > 0$. Þá er til bilaruna $(I_n)_{n \geq 1}$ sem uppfyllir
    \[[a,b] \subseteq \bigcup_{n \geq 1} I_n \text{ og } 0 \sumoinfty \ell(I_n) \leq m^{*}([a,b]) + \eps{2}.\]
    Látum $a_n$ og $b_n$ tákna vinstri og hægri endapunkta bilsins $I_n$ og setjum $J_n := ]a_n - \eps{2}, b_n + \eps{2}[$.
    Þá er $I_n \subseteq J_n$ og $\ell(I_n) = \ell(J_n) - \eps{2^{n+1}}$ og þar með $\sumoinfty \ell(J_n) = \sumoinfty \ell(I_n) + \eps{2}$.
    Af því sést svo að $\sumoinfty \ell(J_n) \leq m^{*}([a,b]) + \epsilon$.
    Nú er $(J_n)_n$ opin þakning á $[a,b]$, svo að til er $n_0$ þ.a. $[a,b] \subseteq J_1 \cup \dotsb \cup J_{n_0}$
    Skv. æfingunni fæst því að $\ell([a,b]) \leq \sum_{n =1}^{n_0} \ell(J_n) < \sumoinfty \ell(J_m) \leq m^{*} ([a,b]) + \epsilon$.

    Gerum nú næst ráð fyrir að $I = ]a,b[$ og tökum eitthvert $\epsilon > 0$. Þá fæst:
    \begin{gather*}
      \ell(]a,b[)  \leq \ell([a + \eps{2}, b - \eps{2}]) + \epsilon \\
      \leq m^{*}([a + \eps{2}, b - \eps{2}]) + \epsilon \\
      \ub{\leq}{2.2.3} m{*}(]a,b[) + \epsilon
    \end{gather*}


    Loks fæst að fyrir $I = [a,b[$ eða $I = ]a,b]$ að
    \[\ell(I) = \ell(]a,b[) \leq m^*(]a,b[) \ub{\leq}{2.2.3} m^*(I)\]
\end{proof}


\begin{proof}[Sönnun á 2.2.5]
  G.r.f. að $\sumoinfty m^*(E_n) < \infty$; annars er ekkert að sanna.

      Gefum okkur $\epsilon > 0$. Fyrir sérhvert $n \geq 1$ er til runa af bilum
      $(I^n_k)_{k \geq 1}$ sem þekja $E_n$ og uppfylla
      $\sumninfty{k} \ell(I^n_k) \leq m^*(E_n) + \eps{2^n}$. Þá fæst

      \begin{gather*}
        m^*(\bigcup_{n=1}^{\infty} E_n) \leq \sum_{(m,n) \in \N^2} \ell(I_k^n)
          \ub{=}{innskot} \sumoinfty (\sumninfty{k} \ell(I^n_k))\\
          \leq \sumoinfty (m^*(E_n) + \eps{2^n}) = \sumoinfty m^*(E_n) + \epsilon
      \end{gather*}
        
\end{proof}
\begin{proof}[Sö á 2.2.6]
     Leiðir beint af því að lengd bils er óháð hliðrun 
\end{proof}

\section{Lebesgue-mælanleg mengi og Lebesgue-málið}

Þetta skilyrði kemur frá Carathéodory.

\begin{ath} 
  \begin{enumerate}[(i)] 
  \item \[m^*:  \cP(\R) \to [0, \infty] \]
    og um sérhvert bil $I$ gildir $m^{*}(I) = \ell(I)$
    með hjálp valfrumsendunnar er unnt að sýna fram á að til sé runa
    $(X_n)_{n\geq1}$ í $\cP(\R)$, sem eru innbyrðis sundurlæg og $m^*(\bigcup_{n\geq1}X_n) < \sumoinfty m^*(X_n)$ (*)
    Viljum því einskorða $ m^* $ við minna safn hlutmengja sem efur tiltekna eiginleika og meðal annars þannig að $=$ gildi
    í stað $<$ í (*) hér að ofan.

  \item Skv. setn. 2.2.5 er ójafnan $m^*(A) \leq m^*(A\cap E) + m^*(A \cap E^C)$.
     
  \end{enumerate}
\end{ath}

\chapter{2015-01-09}

\section{Fyrri fyrirlestur}
\subsection{}
\begin{proof}
  Við vitum að ójafnan \[m^*(A) \leq m^*(A \cap E) + m^*(A \cap E^C) \]
  gildir alltaf.
  
\end{proof}

\subsection{}
\begin{proof}
  \begin{enumerate}[(i)]
  \item  Látum N vera núllmengi og $A \subseteq \R$. Þá fæst
    \begin{gather*}
      m^*(A\cap N) \leq m^*(N) = 0 \text{ vegna } A \cap N \subseteq N \\
      m^*(A \cap N^C) \leq m^*(A) \text{ vegna } A \cap N^C \subseteq A
    \end{gather*}
    og því $m^* \geq m^*(A \cap N) + m^*(A\cap N^C)$.


  \item Látum okkur nægja að skoða bil af gerðinni $I = [a;b]$. Tökum $A \subseteq \R$
    og $\epsilon > 0$. Veljum þakningu $(I_n)_{n\geq 1}$ af bilum fyrir $A$,
    sem uppfyllir $m^*(A) \leq m^*(A) \leq \sum_{k=1}^{\infty} \ell(I_n) \leq m^*(A) + \epsilon$
    Ljóst er að bilin $I_n' := I_n \cap [a,b]$ þekja $A \cap [a,b]$, svo að
    $m^*(A \cap [a,b]) \leq \sum_{k=1}^{\infty} \ell(I_n')$.

    Bilin $I_n'' := I_n \cap ]- \infty, a[$ og $I_n''' := I_n \cap ]b, \infty [ $
    þekja  $A \cap ]a,b[^c$ svo að

\[ m^*(A \cap [a,b]^c) \leq \sum_{k=1}^{\infty} \ell(I_n'') + \sum_{k=1}^{\infty} \ell(I_n''') \]
  \end{enumerate}
  
\end{proof}

\subsection{}

\begin{proof}
  \begin{enumerate}[(i)]
  \item  $\R \in \cM $:

    Ef $A \subseteq \R$, þá er $A \cap \R = A$ og $A \cap \R^C = A \cap \emptyset = \emptyset$,
svo að 
\[ m^*(A) = \underbrace{m^*(A \cap \R)}_{= m^*(A)} +  \underbrace{m^*(A\cap \R^C}_{= 0}\]
\item Ef $E \in \cM$ og $A \subseteq \R$, þá gildir
\[ m^*(A) = m^*(A \cap E) + m^*(A \cap E^C) = m^*(A \cap ( E^C)^C) + m^*(A \cap E^C) \]
svo að $E^C \in \cM$
\item Gerum fyrst ráð fyrir að $(E_k)_{k \geq 1}$ sé runa af innbyrðis sundurlægum mengjum úr $\cM$.

  \begin{itemize}
  \item Byrjum á að sýna með þrepun, að fyrir sérhvert $A \subseteq \R$ gildi:
    \[ m^*(A) = \sum_{k=1}^n m^*(A \cap E_k) + m^*(A \cap (\bigcup_{k=1}^n E_k)^C) \]

    $\mathbf{n = 1}$: \[m^*(A) = m^*(A \cap E_1) + m^*(A \cap E_1^C) \text{ vegna } E_1 \in \cM \]


    $\mathbf{(n-1) \Rightarrow n}$: Látum $A \subseteq \R$. Þar sem $E_n$ er úr $\cM$, þá
    fæst:
    \[ m^*(A \cap (\bigcup_{k=1}^{n-1} E_k)^C) =  m^*(A \cap (\bigcup_{k=1}^{n-1} E_k)^C \cap E_n) +  m^*(A \cap (\bigcup_{k=1}^{n-1} E_k)^C \cap E^C) \]

    Nú er $(\bigcup_{k=1}^{n-1} E_k)^C \cap E_n = E_n$ ( $E_k$-in eru innb. sundurlæg).
    svo umskrifa má síðustu jöfnuna:
      \[ \ub{m^*(A \cap (\bigcup_{k=1}^{n-1} E_k)^C)}{$= m^*(A) - \sum_{k=1}^n m^*(A \cap E_k)$ skv. þf. } = m^*(A \cap E_n) + m^*(A \cap (\bigcup_{k=1}^{n} E_k)^C ) \]
      með því að nota de Morgan,
      og því
      \[ m^*(A) = \underbrace{\sum_{k=1}^{n-1} m^*(A \cap E_k) + m^*(A \cap E_n)}_{\sum_{k=1}^{n} m^*(A\cap E_k)} + m^*(A \cap (\bigcup_{k=1}^{n} E_k)^C) \]
    \item Sýnum nú að $\bigcup_{k \geq 1} E_k \in \cM$ og $m^*(\bigcup_{k \geq 1} E_k) = \sum_{k=1}^{\infty} m^*(E_k)$.

      Þar sem $(\bigcup_{k=1}^{\infty} E_k)^C \subseteq (\bigcup_{k=1}^n E_k)^C$ fyrir öll $n$, þá fæst:
      \[ m^*(A) \geq \sum_{k=1}^n m^*(A \cap E_k) + m^*(A \cap (\bigcup_{k=1}^{\infty} E_k)^C) \: \forall n \in \N. \]
      
      Meeð því að láta $n \to \infty$ fæst því
      \begin{gather*}
        m^*(A) \geq \sum_{k=1}^{\infty} m^*(A \cap E_k) + m^*(A \cap (\bigcup_{k=1}^{\infty} E_k)^C)\\
        \geq m^*(A \cap [\bigcup_{k=1}^{\infty} E_k]) + m^*(A \cap (\bigcup_{k=1}^{\infty} E_k)^C)
      \end{gather*}

      Bæði ójöfnumerkin eru því jafnaðarmerki, svo að $\bigcup_{k=1}^{\infty} E_k \in \cM$  og
      \[ m^*(A) = \sum_{k=1}^{\infty} m^*(A \cap E_k) + m^*(A \cap (\bigcup_{k=1}^{\infty} E_k)^C) \]

Með því að taka $A = \bigcup_{k=1}^{\infty} E_k$ fáum við \[m^*(\bigcup_{k=1}^{\infty} E_k) = \sum_{k=1}^{\infty} m^*(E_k) + \underbrace{m^*(\emptyset)}_{= 0}\]
  \end{itemize}
  \end{enumerate}
\end{proof}
\subsection{}

\begin{enumerate}[\textbf{\arabic*}.]
\item  $\cP(\Omega)$ er $\sigma$-algebra.
\item $\set{\emptyset, \Omega}$ er $\sigma$-algebra.
\item $ \Omega := \N, O = \set{1,3,5,\dotsc}, J = \set{2,4,6,\dotsc}$.
Þá er $\set{\emptyset, O, J,\N}$ $\sigma$-algebra
\item  Látum $(\mathcal{F}_{\alpha})_{\alpha \in \Lambda}$ vera fjölskylda af
$\sigma$-algebrum á $\Omega$, þá er $\bigcap_{\alpha \in \Lambda} \mathcal{F}_{\alpha}$
$\sigma$-algebra á $\lambda$.
\begin{itemize}
\item $\omega \in \cF_{\alpha}$ $\forall \alpha \in \Lambda$, svo að $\Omega \in \bigcap_{\alpha \in \Lambda} \cF_{\alpha}$
\item Ef $E \in \bigcap_{\alpha \in \Lambda} \mathcal{F}_{\alpha}$, þá er $E \in \cF_{\alpha}$ $\forall \alpha$ og
  því $E^C \in \cF_{\alpha}$ $\forall \alpha$ og þar með $E^C \in \bigcap_{\alpha \in \Lambda} \mathcal{F}_{\alpha}$


\item Ef $(E_n)_{n \geq 1}$ er runa í $\bigcap_{\alpha \in \Lambda} \mathcal{F}_{\alpha}$, þá er $(E_n)_{n \geq 1}$
  runa í $\cF_{\alpha}$ $\forall \alpha$ og því $\bigcup_{n \geq 1} E_n \in \cF_{\alpha}$ $\forall \alpha$ og þar með
  $\bigcup_{n\geq 1} E_n \in \cF_{\alpha}$

\item
  \begin{ath}
    
  Ef $\Lambda = \emptyset$, þá er $\bigcap_{\alpha \in \emptyset} \cF_{\alpha}$ = $\cP(\Omega)$ (almengið).
\end{ath}
\end{itemize}

\item Ef $\cS$ er eitthver safn hlutmengja í $\Omega$, þá er sniðmengi allra $\sigma$-algebra á $\Omega$, sem
  innihalda $\cS$ kölluð \emph{$\sigma$-algebran sem $\cS$ framleiðir}.
  Köllsum hana $\cF_{\cS}$. Hún hefur eftirfarandi eiginleika (þ.e.a.s hún er minnst allra
  $\sigma$-algebra sem innihalda $\cS$): Ef $\cF$ er $\sigma$-algebra á $\Omega$ og $\cS \subseteq \cF$, þá
  $\cF_{\cS} \subseteq \cF$


\item  Ef $\Omega$ er firðrúm (grannrúm), þá kallast $\sigma$-algebran, sem opnu mengin framleiða,
  \emph{Borel-algebran} á $\Omega$; hún er einnig framleidd af lokuðum mengjum.
\end{enumerate}

\subsection{}

\begin{daemi}
  \begin{enumerate}[\arabic*.]
  \item $\mu_1, \mu_2$: $\cP(\Omega) \to [0, \infty]$
    \[ \left. \begin{array}{l}
                \mu_1 (E) = 0 \: \forall E \in \cP(\Omega)\\
                \mu_2 (\emptyset) = 0, \mu_2(E) = \infty \Ef E \neq \emptyset \\
                \end{array}
                \right\} \text{ mál á $\cP(\Omega)$}
                \]
              \begin{ath}
                $\mu_1 = \mu_2 \Leftrightarrow \Omega = \emptyset$
              \end{ath}


            \item Tökum punkt $p$ úr $\Omega$. Fallið $\mu: \cP(\Omega) \to [0,\infty]$
sem skilgr. er með \[ \mu(E) := \bcondef 0 & \Ef p \not\in E\\ 1 & \Ef p \in E \econdef \]
er mál á $\cP(\Omega)$.

\item $\mu: \cP(\N) \to [0, \infty], \mu(E) = \#(E)$ er mál, oft kallað \emph{talningarmálið} á $\N$.
              \end{enumerate}

  
\end{daemi}

\section{Seinni fyrirlestur}

Farið var í sönnun á 3.1.3.

\chapter{2015-01-12}

\begin{proof}[Sönnun á setn 3.1.3 (framhald)]
Búin að sanna (i) og (ii). (iii) Búið að sanna. Ef $(E_n)_{n \geq 1}$ er runa í $\cM$
af innbyrðis sundurlægum mengju, þá er $\bigcup_{n \geq 1} E_n \in \cM$.
og $m(\bigcup_{n \geq 1} E_n) = \sumoinfty m (E_n)$

$\cdot$ \emph{Sýna}: sammengi endanlegra margra (ekki endilega sundurlægra) mengja úr $\cM$ sé í $\cM$.
Okkur nægir að skoða tvö mengi.

\emph{Nú}: Fyrir sérhvert $A \subseteq \R$
 gildir að \[m^*(A) = m^*(A \cap E_1) + m^*(A \cap E_1^C)\]
 og \[m^*(A \cap E_1^C)  = m^*(A \cap E_1^C \cap E_2) + \underbrace{m^*(A \cap E_1^C \cap E_2^C}_{m^*(A \cap (E_1 \cup E_2)^C)}\]
 og því
 \begin{gather*}
   m^*(A) = \underbrace{m^*(A \cap E_1) + m^*(A \cap E_1^C \cap
     E_2)}_{\geq m^*(A \cap (E_1 \cup E_2))} + m^*(A \cap (E_1 \cup E_2)^C)\\
    \geq m*(A \cap (E_1 \cup E_2)) + m^*(A \cap (E_1 \cup E_2)^C)
  \end{gather*}
  og því $E_1 \cup E_2 \in \cM$.
  Skv. (ii) fæst því (de morgan) að enandlegt sniðmengi mengja úr $\cM$ er í $\cM$.

  $\cdot$ Lokahnykkur: Látum $(E_k)_{k \geq 1}$ vera runu í $\cM$ og śynum að
  $\bigcup_{k \geq 1} E_k \in \cM$. Mengin $F_1 := E_1$, $F_2 := E_1 \setminus E_1$,
  $F_k := E_k \setminus \underbrace{\bigcup_{j=1}^{k-1} E_j}_{E_k \cap (\bigcup_{j=1}^{k-1} E_j)^C}, \dotsc$ eru í $\cM$.
  og auðséð er að $\bigcup_{k \geq 1} F_k = \bigcup_{k \geq 1} E_k$. Þar með
  er $\bigcup_{k \geq 1} E_k \in \cM$ vegna þess að $F_k$-in eru innbyrðis sundurlæg.
\end{proof}

\section{Fyrri fyrirlestur}
\subsection{}
\begin{proof}
  \begin{enumerate}[(i)]
  \item \[ m(A) = m^*(A) \leq m^*(B) = m(B) \]
  \item Vitum skv. síðustu sönnun að endanl. sammengi af mengjum úr $\cM$ eru
    í $\cM$.
    Nú: Ef $A \subseteq B$, þá $B \setminus A = B \cap A^C \in \cM$
    og $B = A \cup (B \setminus A)$ með $A \cap (B \setminus A) = \emptyset$.
    Þar með er $m(B) = m(A) + m(B \setminus A)$ og því
    $m(B \setminus A) = m(B) - m(A)$.

  \item Leiðir beint af því að $m^*$ er óháð hliðrun.
  \end{enumerate}
\end{proof}

\subsection{}
\begin{proof}
  $m^*(A \Delta B) = 0$ hefur í för með sér að $B \setminus A$ og $B \setminus A$
  eru bæði núllmengi og þar með bæði í $\cM$. Af því leiðir að mengin
  $ A \cap B = A \setminus (A \setminus B) $ og $B = B \setminus (B \setminus A)$
  eru í $\cM$.
  Við fáum því
  \begin{gather*}
    m(A) \underbrace{=}_{A = (A \cap B) \amalg (A \setminus B)} m( A \cap
    B) + \underbrace{m(A \setminus B)}_{= 0}\\
    = m(A \cap B) + \overbrace{m(B \setminus A)}^{= 0} \underbrace{=}_{B = (A \cap B) \amalg
      (B \setminus A)} m(B)
  \end{gather*}
\end{proof}

\subsection{}

\begin{proof}
  \begin{enumerate}[(i)]
  \item Til er runa af \emph{opnum} bilum $(I_n)_{n \geq 1}$ þ.a.
    $A \subseteq \bigcup_{n \geq 1} I_n$ og
    \[ \sum_{n \geq 1} \underbrace{m(I_n)}_{= \ell(I_n)} \leq m^*(A) + \varepsilon \]
    Setjum $O := \bigcup_{n \geq 1} I_n$ og fáum að O er opið mengi og
    $ m(O) \leq \sum_{n \geq 1} m(I_n) \leq m^*(A) + \varepsilon$
  \item Gerum fyrst ráð fyrir að $m(A) < \infty$ og veljum O eins og í (i) með
    $\f{\varepsilon}{2}$ í stað $\varepsilon$.
    Þá fæst að $m(O \setminus A) = m(O) - m(A) < \varepsilon$

    Í tilfellinu $m(A) = \infty$ setjum við, fyrir sérhvert $n \geq 1$,
    $A_n := A \cap [-n,n]$.
    Þar sem $m(A_n) \leq 2 n \: \forall n$
    þá er til opið $O_n$ þ.a. $A_n \subseteq O_n$
    og $m(O_n \setminus A_n) \leq \f{\varepsilon}{2^n}$.
    Þá er $O := \bigcup_{n \geq 1} O_n$ opið.
    $A \subseteq O$ og 
\[O \setminus A = \bigcup_{n \geq 1} O_n \setminus \bigcup_{n \geq 1} A_n \subseteq_{n \geq 1} (O_n \setminus A_n)\]
Þar með fæst \[ m(O \setminus A) \leq \sumoinfty m(O_n \setminus A_n) \leq \varepsilon \]
    
    
  \end{enumerate}
\end{proof}

\section{Seinni fyrirlestur}







 
\end{document}

