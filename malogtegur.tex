\documentclass[12pt]{book}
\usepackage[utf8]{inputenc}

% not needed with polyglossia
\usepackage[utf8]{inputenc}
\usepackage[T1]{fontenc}

%xe/lualatex
%\usepackage{polyglossia}
%\setdefaultlanguage{icelandic}


\usepackage{graphics,amsmath,amsfonts,amsbsy,amssymb,amsthm}
\usepackage{fancyvrb}
\usepackage[a4paper]{geometry}
\usepackage{graphicx}
\usepackage{hyperref}
\usepackage{datatool}
\usepackage{float}
\usepackage{mdframed}
\usepackage{listingsutf8}
\usepackage{enumerate}
\usepackage{comment}
\usepackage{epstopdf}
\usepackage{caption}
\usepackage{subcaption}
\usepackage{titling}
\usepackage{tikz}
\usepackage[shortlabels]{enumitem}
\usepackage{mathtools}
\usepackage{tabu}

\usepackage{accents}

\setlength{\parskip}{8pt plus 1pt minus 1pt}
%Verdur ad vera her, sumir pakkar dependa a thetta.
\usepackage[icelandic]{babel}

\newcommand{\subtitle}[1]{%
  \posttitle{%
    \par\end{center}
    \begin{center}\large#1\end{center}
    \vskip0.5em}%
}
%viljum ekki númeraða kafla á dæmum



\newcommand{\nonums}{\setcounter{secnumdepth}{-1}}

%flýtiskipanir
\newcommand{\e}{\emph}


%\newcommand{\R}{\Real}
%\newcommand{\C}{\Complex}
%\newcommand{\Z}{\Integer}
%\newcommand{\N}{\Natural}
%\newcommand{\Q}{\Rational}
\newcommand{\R}{\mathbb{R}}
\newcommand{\X}{\mathbb{X}}
\newcommand{\Y}{\mathbb{Y}}
\newcommand{\K}{\mathbb{K}}
\newcommand{\C}{\mathbb{C}}
\newcommand{\Con}{\mathcal{C}}
\newcommand{\Z}{\mathbb{Z}}
\newcommand{\N}{\mathbb{N}}
\newcommand{\Q}{\mathbb{Q}}
\newcommand{\f}{\frac}
\newcommand{\1}{\frac{1}}
\newcommand{\eps}{\f{\epsilon}}
\newcommand{\Lra}{\Leftrightarrow}
\newcommand{\Th}{\text{ þegar }}
\newcommand{\Ef}{\text{ ef }}
\newcommand{\Og}{\text{ og }}


\newcommand{\inner}[1]{\accentset{\circ}{#1}}
\newcommand{\eR}{\widetilde{\R}}

\newcommand{\sumninfty}[1]{\sum_{n = {#1}}^{\infty}}
\newcommand{\sumoinfty}{\sum_{n = 1}^{\infty}}
\newcommand{\sumzinfty}{\sum_{n = 0}^{\infty}}

\newcommand{\com}[1]{\set{\text{#1}}}
\newcommand{\Com}[1]{\set{\text{Athsmd: \text{#1}}}}

\newcommand{\ub}[2]{\underbrace{#1}_{\text{#2}}}
\newcommand{\ubt}[2]{$\ub{\text{#1}}{#2}$}


\newenvironment{inum}{\begin{enumerate}[label=(\roman*).]}{\end{enumerate}}
\newenvironment{anum}{\begin{enumerate}[label=(\alph*).]}{\end{enumerate}}


\newcommand{\bcondef}{\left\{ \begin{array}{l l}}
\newcommand{\abs}[1]{\left|#1\right|}

\newcommand{\econdef}{\end{array} \right.}
\DeclarePairedDelimiter{\condef}{\bcondef}{\econdef}

\DeclarePairedDelimiter{\ceil}{\lceil}{\rceil}
\DeclarePairedDelimiter{\floor}{\lfloor}{\rfloor}
\DeclarePairedDelimiter{\set}{\{}{\}}
\DeclarePairedDelimiter{\braket}{\langle}{\rangle}


\newenvironment{lausn}{\begin{proof}[Lausn]}{\end{proof}}

\newcommand{\sep}{\;|\;}

\newcommand{\fig}[2]{
\begin{figure}[H]
  \centering
  \includegraphics{#1}
  \caption{#2}
  \label{fig:#1}
\end{figure}
}


\newtheorem*{setn}{Setning}
\newtheorem*{hsetn}{Hjálparsetning}
\lstset{  literate={á}{{\'a}}1
                  {ó}{{\'o}}1
                  {ú}{{\'u}}1
                  {ð}{{\dh}}1
                  {í}{{\'i}}1
                  {é}{{\'e}}1
                  {ö}{{\"o}}1
                  {þ}{{\th}}1
                  {æ}{{\ae}}1
                  {ý}{{\'y}}1
                  {Á}{{\'A}}1
                  {Ó}{{\'O}}1
                  {Ú}{{\'U}}1
                  {Ð}{{\DH}}1
                  {Í}{{\'I}}1
                  {É}{{\'E}}1
                  {Ö}{{\"O}}1
                  {Þ}{{\TH}}1
                  {Æ}{{\AE}}1
                  {Ý}{{\'Y}}1}


\theoremstyle{definition}
\newtheorem*{skgr}{Skilgreining}
\newtheorem*{daemi}{Dæmi}
\newtheorem*{frumsenda}{Frumsenda}

\theoremstyle{remark}
\newtheorem*{ath}{Athugasemd}
\newtheorem*{innsk}{Innskot}

\title{Tegur- og Málfræði}
\subtitle{Fyrirlestrarnótur}
\author{Matthías Páll Gissurarson}
\date{Vor 2015}


\begin{document}

\maketitle

\tableofcontents

\section{Mengi: Ritháttur og upprifjun}
Gefum okkur að til grundvallar liggi ``hæfilega''  stórt almengi.

\begin{itemize}
\item Fjölskylda af hlutmengjum í mengi M er vörpun \(a: \Lambda \to \mathcal{P}(M) \). Skrifum í stað $a(\alpha)$ og táknum $a$ með $(A_{\alpha})_{\alpha \in \Lambda}$. $\Lambda$ kallast \emph{stikamengi} fjölskyldunnar.
\item \[ \bigcap_{\alpha \in \Lambda} := \set{x \in M | x \in A_{\alpha} \text{ f. \emph{öll} $\alpha$ úr $\Lambda$}}\]
\item \[ \bigcup_{\alpha \in \Lambda} := \set{x \in M | x \in A_{\alpha} \text{ f. \emph{eitthvert} $\alpha$ úr $\Lambda$}}\]

\item Fyrir $A \subseteq M$ setjum við \[A^C = M \setminus A := \set{ x \in M | x \not\in A}\] og köllum \emph{fyllimengi} A (í M).
\item Fyrir $ A, B \subseteq M$ setjum við \[B \setminus A := \set{ x \in B | x \not\in A} = B \cap A^C\]
\item Fyrir $A,B \subseteq M$ kallast \[ A \Delta B := ( A \setminus B) \cup (B \setminus A) \] (s.s. bara þau stök sem eru í öðru hvoru, en ekki báðum) \emph{samhverfur mismunur} A og B.

\item \emph{Reglur de Morgan} \[ (\bigcup_{\alpha \in \Lambda} A_{\alpha})^C = \bigcap_{\alpha \in \Lambda} A_{\alpha}^C\]

\[ (\bigcap_{\alpha \in \Lambda} A_{\alpha})^C = \bigcup_{\alpha \in \Lambda} A_{\alpha}^C\]
\item Fyrir $ A \subseteq M$ skgr. við kennifall $A$:
\[ 1_A: M \to \R, 1_A(x) = \bcondef 1 & \Ef x \in A \\ 0 & \Ef x \not\in A \econdef \]A

\item Fyrir $A,B \subseteq M$ gildir $1_{A \cap B} = 1_A \cdot 1_B$, $1_{A \cup B} = 1_A + 1_B - 1_{A \cap B}$ og $1_{A^C} = 1 - 1_A$.
\end{itemize}

\section{Firðrúm}
Hlutmengi í firðrúmi er opið \emph{þþaa} það sé sammengi af opnum kúlum. Sér í lagi er hlutmengi í $\R$ opið \emph{þþaa} það sé sammengi af opnum bilum.

\chapter{}
\section{Riemann Darboux- heildið}
er mjög þunglamalegt og hegðar sér illa m.t.t markgilda; runa af Riemann heildanlegum föllum getur hæglega stefnt (í sérhverjum punkti) á fall sem er ekki Riemann heildanlegt (stundum kallað einfaldlega samleitið).

\begin{daemi}
\[f: [0,1] \to R, f(x) = \bcondef 0 & \Ef x \not\in \Q \\ 1 & \Ef x \in \Q \econdef  | f = 1_{Q \cap [0,1]}. \]

$\Q$ er teljanlegt, svo að til er gagntækt fall $q: \N \to \Q \cap [0,1]$, þ.e.a.s. til er runa $(q_n)_{n \in \N}$ þar sem sérhver ræð tala úr $[0,1]$ kemur nkvl. einu sinni fyrir. Setjum $A_n = \set{q_1, \dotsc, q_n}$ og $f_n = 1_{A_n}$. Þá er $f_n \nearrow f$ og ljóst er að $f_n$-in eru Riemann heildanleg, en $f$ ekki.

Hugmyndin er sú að stækka flokk heildanlegra falla þ.a. ekki $f$ gildi.

Allar eðlilegar reiknireglur varðandi heildi gilda áfram.

Flokkurinn er mun þjálli m.t.t. ýmissa aðgerða, sérstaklega markgildistöku.
\end{daemi}

\section{Núllmengi}
\subsection{Dæmi}
\begin{enumerate}[i)]
\item Öll teljanleg hlutmengi í $\R$ eru Núllmengi: Ef $(x_n)_{n \geq 1}$ er upptalning á stökum teljanlegs mengis, þá er $A = \cup_{n \geq 1} [x_n, x_n]$ og $\sum_{n = 1}^{\infty} l([x_n,x_n]) = \sum_{n=1}^{\infty} 0 = 0$
\item $l(\emptyset) = l(]0,0[) = 0$, svo að $\emptyset$ er núllmengi.
\end{enumerate}

\section*{Innskot}

\begin{setn}
Látum $\sum_{n=1}^{\infty}a_n$ vera alsamleitna tvinntalnaröð og $\sigma: \overbrace{\N}^{n \geq 1} \to \N$ vera gagntæka vörpun (umröðun). Þá er röðin $\sumninfty{1} a_{\sigma(n)}$ alsamleitin og $\sumninfty{1} a_{\sigma(n)} = \sumoinfty a_n$.
\end{setn}
\begin{proof}
Fyrir $N > 0$ er til $M > 0$ þ.a. $\set{\sigma(1), \dotsc, \sigma(n)} \subseteq \set{1, \dotsc, M}$.
\[\sum_{n=1}^{N}  \abs{a_{\sigma(n)}}
\leq \sum_{n = 1}^M \abs{a_n}
\ub{\leq \sumninfty{1} \abs{a_n}}{$< +\infty$}
\]
þar með er $\sumoinfty \abs{a_{\sigma(n)}}$ alsamleitin.
Setjum nú fyrir öll $n \geq 1$ $S_n := \sum_{k=1}^n a_k$ og $T_n := \sum_{k=1}^n a_{\sigma(k)}$.
Okkur nægir að sýna að f. sérhvert $\epsilon > 0$ sé til $n_0$ þ.a. $\abs{S_n - T_n} \leq \epsilon$ f. öll $n \geq n_0$.

Gefum okkur $\epsilon > 0$ og veljum $N$ þ.a. $\sum_{j = N+1}^{\infty} \abs{a_j} \leq \epsilon$. Tökum svo $n_0$ þ.a. $ \set{1, \dotsc, n} \subseteq \set{\sigma(1), \dotsc, \sigma(n_0)}$. Þá styttast tölurnar $\abs{a_n, \dotsc, a_n}$ allar út í mismuninum $S_n - T_n$, og þar með $\abs{S_n - T_n} \leq \sum_{j = N+1}^{\infty} \abs{a_j}  \leq \epsilon$ ef $n \geq n_0$.
\end{proof}

\begin{setn}[Umröðunarsetning Riemanns]
Ef $\sum a_n$, þar sem $a_n$ eru rauntölur er samleitin, en $\sum \abs{a_n} = \infty$, þ.e.
samleitin en ekki alsamleitin, og $L \in [-\infty, +\infty]$, þá er til gagntæk vörpun
$\sigma: \N \to \N$ sem hefur þann eiginleika að $\sumoinfty a_{\sigma(n)} = L$.
\end{setn}
\begin{proof} Sleppt.

Lausleg útskýring:
\[a_n^+ \bcondef a_n & \Ef a_n \geq 0 \\ 0 & \Ef a_n < 0 \econdef 
a_n^- \bcondef \abs{a_n} & \Ef a_n \leq 0 \\ 0 & \Ef a_n > 0 \econdef
\]
Ef $a_n$ er alsamleitin, þá er $\sum a_n = \ub{\sum a_n^+}{saml} - \ub{\sum a_n^-}{saml}$,
en ef $\sum a_n$ er samleitin en ekki alsamleitin, þá eru $\sum a_n^+$ og $\sum a_n^-$ báðar ósamleitnar.

\end{proof}

Látum nú $\Lambda$ vera eitthvað teljanlegt mengi og $(a_{\alpha})_{\alpha \in \Lambda}$ vera fjölskylda af tvinntölum. Við segjum að summan $\sum_{\alpha \in \Lambda}$ sé alsamleitin ef
til er gagntæk vörpun $\sigma: \N \to \Lambda$ þ.a. $\sumoinfty \abs{a_{\sigma(n)}} < +\infty$ (þ.e.a.s. $\sumoinfty \abs{a_{\sigma(n)}}$ er alsamleitin). Táknum þá summuna með $\sumoinfty \abs{a_n}$. Skv. setn sem við vorum að sanna, þá er þessi eiginleiki óháður valinu á $\sigma$.


\section*{Æfing}

Látum $(a_{\alpha})_{\alpha \in \Lambda}$ eins og áður. Sýnið: Ef $a_{\alpha}$ er alsamleitin, þá gildir f. sérhvert $\epsilon > 0$ að til er endanlegt hlutmengi $I$ í $\Lambda$ sem uppfyllir
\[ \abs{\sum_{\alpha \in \Lambda} a_{\alpha} - \sum_{\alpha \in J} } < \epsilon \]
fyrir öll endanleg $J$ þ.a. $I \subseteq J \subseteq \Lambda $. 

\begin{ath} Út frá umröðunarsetningu fæst að samleitin röð $\sumoinfty a_n$ sem breytist ekki við umraðanir er alsamleitin. 

%{} skilgreinir lokun, þ.a. inní henni get ég deffað eins og ég vil.
{ 
    \def\aal{a_{\alpha}}
    \def\aall{(\aal)_{\alpha \in \Lambda}}
    \def\saal{\sum_{\alpha \in \Lambda} a_{\alpha}} 

\newcommand{\sal}[1]{\sum_{\alpha \in #1} a_{\alpha } }

Takið eftir að summa 
$\saal$ er alsamleitin þþaa til sé $ K > 0$ sem uppfyllir

$\sal{I} \abs{\aal} \leq K$ f. öll endanleg $I \subseteq \Lambda$.

\def\mn{(m,n)}
\def\amn{a_{m,n}}
Lítum nú á tilfellið $(\amn)_{\mn \in \underbrace{\N \times \N}_{\N^2}}$.

Ef $\sum_{\mn \in \N^2} \amn$ er alsamleitin, þá er til $k > 0$ þ.a.
$\sum_{m=1}^M \sum_{n=1}^N | \amn| \leq k$ f. öll $M$ og $N$.

Þá er $\sum_{n=1}^N \abs{\amn} \leq k$ f. öll m og öll $N$. og því
$\sumoinfty \abs{\amn} \leq k$ og $\forall m$ og þar með $\sum_{n=1}^N \left(\sum_{m=1}^M \abs{\amn}\right) \leq k$ f. öll $M$. Af því leiðir að $\sum_{m=1}^{\infty} \left( \sumoinfty \abs{\amn} \right) \leq k$ og þar með er $\sum_{m=1}^{\infty} \left( \sumoinfty \abs{\amn} \right)$ alsamleitin.
}

\end{ath}
\chapter{?}

\section*{Innskot}

{

\def\mn{(m,n)}
\def\amn{a_{m,n}}
\def\summn{\sum_{\mn \in \N^2}}

... % Hér vantar fyrri hluta fyrsta tíma.
Þá gildir um öll $M$ og $N$ að $\sum_{m=1}^{\infty} \sum_{n=1}^{\infty} \abs{\amn} \leq \summinfty (\sumoinfty \abs{\amn})$, svo að $\sum_{\mn \in \N^2} \amn$ er alsamleitin.

Sýnum að þá sé $\summinfty ( \sumoinfty \amn ) = \summn \amn$.

Nú: Gefum okkur $\epsilon > 0$. Veljum M svo stórt að $\summinfty ( \sumoinfty \amn ) < \eps{4}$.

Veljum svo $N$ það stórt að

... % vantar smá af einni töflu


}

\section{}

\subsection{}
\begin{proof}

Gefum okkur $\epsilon > 0$. Fyrir hvert m veljum við runu $(I^n_k)_{k \geq 1}$ af bilum sem uppfyllir

\[N_n \leq \bigcup_{k \geq 1} I_k^n \text{ og } \sumninfty{k} l(I^n_k) < \eps{2^n}\]

Veljum gagntæka vörpun $\sigma: \N \to \N^2$ og skrifum $(I^n_k)_{(m,n) \in \N^2}$ sem runu
$(J_j)_{j \in \N}$ með hjálp $\sigma$, þ.e.a.s $J_j = I^n_k$

þþaa $\sigma (j) = (m,n)$.

\begin{ath}
$N \subseteq \R$ er núllmengi ef fyrir sérhvert $\epsilon > 0$ er til runa $(I_n)_{n}$ af bilum þ.a.
$N \subseteq \bigcup I_n$ og $\sum_{n \geq 1} l(I_n) < \epsilon$.

\end{ath}

Ljóst er að $N = \bigcup_{n \geq 1} N_n \subseteq \bigcup_{(m,n) \in \N^2} I^n_k = \bigcup_{j \geq 1} J_j$
og skv. innskotinu hér að framan fæst þá að
$\sumninfty{j} l(J_j) = \sumoinfty ( \sumninfty{k} l(I_k^n)) < \sumninfty{n} \eps{2^n} = \epsilon$

\end{proof}

\begin{daemi}

Öll teljanleg hlutmengi í $\R$ eru núll mengi, en Cantor-mengið er dæmi um óteljanlegt núllmengi (sjá bls. 19 í bók og ``mengi og firðrúm'').
\end{daemi}

\section{}
\subsection{}
\subsection{}
\begin{proof}
\end{proof}

\end{document}
